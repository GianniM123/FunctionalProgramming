\documentclass[12pt,a4paper]{report}
\usepackage[utf8]{inputenc}
\usepackage{amsmath}
\usepackage{amsfonts}
\usepackage{amssymb}
\begin{document}


\textbf{Name:} Gianni Monteban \\
\textbf{Studentnumber:} s1047546

\section{Exercise class}
\begin{verbatim}
f1 (a, b) -> a
f1 (x,y) = x

f2 (a,a) -> (a,a)    all same datatype
f2 (x,y) -> (x,y)

f3 (a,b) -> (b,a)
f3 (x,y) -> (y,x)

f4 (a -> b)-> a -> b
f4 f x = f (x)

f5 (a,c) -> a
f5 (x,y) = x

f6 (c -> a -> b, a, c) -> b
f6 (f, a, c) = f (c) (a)

f7 (a -> b, c -> a, c) -> b
f7 (f,g,c) = f(g (c))

f8 (c->a->b, c->a, c) -> b
f8 (f,g,c) = f (c) (g(c))

f9 Int -> (Int -> Int)
f9 x y = x + y

f10 (Int -> Int) -> Int
f10 f = f 1

f11 a -> (a -> a)
f11 x y = x
f11 x = \y -> y

f12 (a -> a) -> a
f12 f = 
can't


\end{verbatim}

\section{Exercise 2.3}
The Int is overflowing at a certain point. \\
The integer can be unlimited long. The int is limited at $2^{63} - 1$ after this value the int will overflow. When the number gets to large the int will take a modulo of $2^{64}$ and save the module as the value. After $66!$ the modulo of the value is always 0, this means the Int will always be 0.


\section{Exercise 2.7}
\begin{verbatim}


f1 x = x
f1 :: a -> a 

f3 x = (snd x, fst x)  
f3 :: (a,b) -> (b,a)

f6 x y = (x,(y,y),x)
f6 :: a -> b -> (a,(b,b),a)
\end{verbatim}

\section{Exercise 2.8}

See .hs file
\end{document}